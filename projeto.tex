\documentclass{article}

\usepackage[utf8]{inputenc}
\usepackage[brazil]{babel}

\usepackage{bera}
\usepackage{pdflscape}
\usepackage{gantt} % http://www.martin-kumm.de/tex_gantt_package.php

\usepackage{colortbl}
\usepackage{xspace}
\newcommand{\sen}{\mathrm{sen}\,\xspace}



\newcommand\blfootnote[1]{%
  \begingroup
  \renewcommand\thefootnote{}\footnote{#1}%
  \addtocounter{footnote}{-1}%
  \endgroup
}


\usepackage{listingsutf8}
\lstdefinelanguage{pseudocode}
 {morekeywords={se,senao,enquanto,repita,ate,para,retorne},
  %keywordstyle=\fontfamily{fvm}\fontseries{b}\fontshape{b}\selectfont,
  keywordstyle=\fontfamily{fvs}\fontseries{b}\selectfont\bfseries,
  basicstyle=\footnotesize\fontfamily{fvs}\fontshape{n}\selectfont,% \footnotesize was before \fontfamily
  commentstyle=\fontfamily{fvs}\selectfont\slshape,%
  morecomment=[l][commentstyle]{//},
  morecomment=[s][commentstyle]{{/*}{*/}},
  sensitive=false,%
  inputencoding=utf8,
literate={á}{{\'a}}1
         {Á}{{\'A}}1
         {é}{{\'e}}1
         {í}{{\'i}}1
         {Í}{{\'I}}1
         {ó}{{\'o}}1
         {ú}{{\'u}}1
         {à}{{\'a}}1
         {À}{{\`a}}1
         {ã}{{\~a}}1
         {â}{{\^a}}1
         {ê}{{\^e}}1
         {õ}{{\~o}}1
         {ô}{{\^o}}1
         {ç}{{\c{c}}}1,
  string=[d]"
}[keywords,comments,strings]

\lstset{%
%backgroundcolor=\color{Gray},% /usr/share/texmf-texlive/tex/generic/dvips/colordvi.sty
showstringspaces=false,%
%%basicstyle=\fontfamily{cmtt}\fontshape{n}\selectfont,% \footnotesize was before \fontfamily
%%commentstyle=\fontfamily{cmtt}\fontshape{it}\selectfont,%
frame=none,%single,%
texcl=true,%
%extendedchars=\true,
inputencoding=utf8,
language=pseudocode
}
\makeatletter
\define@key{MyCode}{width}[\linewidth]{\def\width{#1}}
\define@key{MyCode}{title}[]{\def\algotitle{#1}}
\define@choicekey*+{MyCode}{type}[\val\nr]{proc,func,algo,none}[none]{%
\ifcase\nr\relax
\def\type{Procedimento}
\or
\def\type{Função}
\or
\def\type{Algoritmo}
\or
{}
\fi
}{\@latex@error{Erroneous value for a choice key}
  {Invalid value '#1' for key 'type' of family Mycode}%
}
\define@key{MyCode}{lstoptions}[]{\def\lstoptions{#1}}
\presetkeys{MyCode}{width,type,lstoptions}{}
\makeatother

\lstnewenvironment{pseudo}[1][]%
{  \setkeys{MyCode}{#1}
   \noindent
   \minipage[t]{\width}
   \vspace{0cm}
   %\noindent\rule[0.4ex]{\linewidth}{1pt}
   \hrule
   \ifdefined\algotitle
     \vspace{0.2\baselineskip}
     \ifdefined\type\relax
           {\bfseries \type}
     \fi
     \algotitle
     \vspace{0.2\baselineskip}
     \hrule
   \fi
   \vspace{0.2\baselineskip}
   \lstset{language=pseudocode,frame=none,mathescape=true,\lstoptions}}
{\hrule\vspace{0.5\baselineskip} %\noindent\rule[0.1ex]{\linewidth}{1pt}
 \vfill\endminipage}


\usepackage{pgf,tikz}
\usepackage{tikz}

%%% http://tex.stackexchange.com/questions/60262/how-to-use-tikz-graph-and-tkz-qtree-without-conflict/60266#60266
%%%
\usepackage{tkz-graph}
%%%%%%%%%%%%%%%%%%%%%%%%%%%%%%
% patch-tkz-graph.tex : patch for tkz-graph

\makeatletter
\renewcommand*{\Edge}[1][]{\tkz@edge[#1]}%
\def\tkz@edge[#1](#2)(#3){%
\setkeys[GR]{edge}{#1}%
 \begingroup%
\ifthenelse{\equal{\cmdGR@edge@double}{}}{%
\tikzset{LocalEdgeStyle/.style={color = \cmdGR@edge@color,
                                line width = \cmdGR@edge@lw}}}{%
\tikzset{LocalEdgeStyle/.style={line width = \cmdGR@edge@dd,
                                color = \cmdGR@edge@double,
                                double distance = \cmdGR@edge@lw,
                                double  = \cmdGR@edge@color}}}%
\ifGR@edge@local%
      \tikzset{EdgeStyle/.style={}}%
      \fi
   \ifthenelse{\equal{\cmdGR@edge@label}{}}{%
     \protected@edef\@tempa{%
     \noexpand   \draw[LocalEdgeStyle,\cmdGR@edge@style,EdgeStyle]}%
                 \@tempa (#2) to (#3)}{%
     \protected@edef\@tempa{%
     \noexpand   \draw[LocalEdgeStyle,\cmdGR@edge@style,EdgeStyle] (#2) to%
    node[fill = \cmdGR@edge@labelcolor,
         text = \cmdGR@edge@labeltext,
         \cmdGR@edge@labelstyle,LabelStyle]}\@tempa
   {\cmdGR@edge@label} (#3)}%
   ;
\endgroup%
}%    
\makeatother  
\endinput

%%%%%%%%%%%%%%%%%%%%%%%%

\usepackage{tikz-qtree}
%%%
%%%

\usepackage[shadow%,disable,%
]{todonotes}

\usetikzlibrary{shapes,fit,backgrounds,positioning,arrows,matrix,shadows,calc}
\tikzstyle{stdarrow}=[->,-triangle 45]
\tikzset{nonode/.style={rectangle,minimum size=1mm}}
\tikzset{stnode/.style={rectangle, draw, thick,color=blue!40,
    minimum size=0.8cm,top color= white,bottom color=blue!40,
    drop shadow,
    align=center,rounded corners=1mm}}

\tikzset{ast/.style={circle, draw, thick,
    minimum size=0.8cm,
    inner color=white,
    outer color=blue!40,
    %fill=blue!25,
    %drop shadow,
    align=center}}
\tikzset{ast2/.style={rectangle, draw, thick,
    rounded corners=0.5mm,
    minimum size=0.8cm,
    inner color=white,
    outer color=blue!40,
    %fill=blue!25,
    %drop shadow,
    align=center}}

\tikzset{manager/.style={rectangle, draw, very thick,
    minimum size=1cm,minimum width=2cm,%fill=red,
    top color=white,bottom color=red!80,
    drop shadow,
    align=center,rounded corners=1mm}}
\tikzset{worker/.style={rectangle, draw, very thick,
    minimum size=1cm,minimum width=2cm,%fill=brown,
    top color=white,bottom color=blue!40,
    drop shadow,
    align=center,rounded corners=1mm}}
 

\newcommand{\ledge}[1]{%
  \draw[-,dotted] (#1.north west) to (#1.north);
  \draw[-,dotted] (#1.north west) to (#1.south west);
}


\usepackage{csquotes}
\usepackage[style=alphabetic,
            backend=biber]{biblatex}
\addbibresource{projeto.bib}



\title{Linguagem {\em mark-up} para dados geofísicos históricos}
\author{Mirelle Alves de Freitas (RA: )\\
  Orientador: Jerônimo Cordoni Pellegrini}
%  Co-orientador: Marcelo Bianchi (IAG/USP)}

\begin{document}

\maketitle

\begin{abstract}
  Desde 1915 estações brasileiras operaram gerando dados
  em papel -- chamados de {\em magnetogramas} e {\em sismogramas}. Os
  dados contidos nestes registros não podem ser usados facilmente para
  análise, porque não estão sequer completamente digitalizados: há um
  processo de 
  digitalização em andamento, e espera-se que o acervo completo seja
  transformado em imagens digitais.
  Este projeto tem como objetivos (i) propor uma estrutura de dados para
  representação de anotações em imagens de dados geofísicos; (ii)
  implementar uma biblioteca para leitura e gravação desses dados;
  (iii) implementação de um algoritmo para detecção de eventos nessas
  imagens, gerando anotações automaticamente; e (iv)  implementar uma
  interface simples para visualização dos metadados.
\end{abstract}


\begin{center}
  \begin{minipage}{10cm}
    \begin{center}
      \textbf{Palavras-chave:} {\em mark-up}; processamento de imagens
    \end{center}
  \end{minipage}
\end{center}

\section{Introdução}

No Brasil, estações sismológicas operaram entre 1976 e 1955, e
observatórios magnéticos entre 1915 e 2007, produzindo uma grande
quantidade de sismogramas e magnetogramas em papel. Esses registros, 
além de estarem sofrendo efeitos deteriorantes do tempo, não podem ser
usados para análise por métodos modernos, uma vez que não se encontram
em formato digital.

A digitalização dos registros permitiria XXXXX.



\section{Metodologia de trabalho}



\section{Objetivos e metas}

O objetivo científico do projeto é contribuir para a preservação do
registro histórico dos sismogramas e magnetogramas, além de
dar um passo inicial para a futura análise desses dados, na forma de
software voltado para a representação e visualização deles.

Para a formação da aluna, o objetivo é exatamente o do programa PDPD;
especificamente, esperamos que a aluna tome contato com a pesquisa
em geofísica, através de sua inserção em um projeto {\em real} e
{\em relevante}, e que desenvolva a capacidade de estudo e elaboração
autônoma de soluções em computação aplicada.

As metas são:
\begin{itemize}
\item[i) ] {\bf especificar} uma estrutura para representar metadados de anotações feitas em imagens digitalizadas
  de sismogramas e magnetogramas;
\item[ii) ] implementar uma {\bf biblioteca} para incluir e recuperar os
  dados na estrutura definida;
\item[iii) ] implementar {\em um} {\bf algoritmo} para tentar identificar
  automaticamente alguma característica das imagens que seja
  interessante do ponto de vista da geofísica;
\item[iv) ] implementar {\bf interface} de visualização para {\em parte}
  dos metadados, usando a biblioteca.
\end{itemize}

Não há compromisso em obter resultados de alta precisão no desempenho
do algoritmo mencionado ma meta (iii), dado que a aluna é ingressante
no BC\&T -- o objetivo desta meta é principalmente o de familiarizar
a aluna com o trabalho de pesquisa na área.
A meta (iv) poderá ser implementada de maneira simplificada, e para
apenas parte das anotações (possivelmente só um campo).

Todo o software produzido será disponibilizado com licença livre.

\section{Cronograma}

A tabela a seguir mostra o cronograma de atividades para a aluna. Além
das metas já descritas, há duas atividades de revisão bibliográfica: a
primeira com o foco em estruturas para armazenamento de metadados de
imagens, e a segunda com o foco em algoritmos para processamento de
imagens.

\begin{center}
\scalebox{0.85}{
\begin{gantt}[fontsize=\small,titlefontsize=\small]{7}{10}
\begin{ganttitle}
  \numtitle{1}{1}{10}{1}
\end{ganttitle}
\ganttbar[pattern=none,fill=blue!40]{revisão biblio}{0}{1}
\ganttbar[pattern=none,fill=blue!40]{especificação}{1}{1}
\ganttbar[pattern=none,fill=blue!40]{biblioteca}{2}{2}
\ganttbar[pattern=none,fill=blue!40]{revisão biblio}{4}{3}
\ganttbar[pattern=none,fill=blue!40]{algoritmo}{5}{3}
\ganttbar[pattern=none,fill=blue!40]{interface}{8}{2}
\end{gantt}
}
\end{center}


\section{Adequação do projeto à aluna}

Salientamos que, embora ingressante na UFABC, a aluna já sabe
programar e tem prática em desenvolvimento Web.

\printbibliography

\blfootnote{Texto e figuras produzidos integralmente em \LaTeX.}

\end{document}

